\documentclass[10pt,twocolumn]{article}
\usepackage{graphicx}
\usepackage{amssymb}
\usepackage{titling}
\usepackage{url,hyperref}

\begin{document}

\title{Machine Learning Assignment 2}

\author{Michelle Ho and Shay Lehmann}

\date{%
CUSP-GX-5006\\
Assignment \# 1\\
\rule{\textwidth}{1pt}
}

\posttitle{\par\rule{3in}{0.4pt}\end{center}\vskip 0.5em}
%\postdate{\rule{\textwidth}{1pt}}

\maketitle

\begin{abstract}
In this assignment, the authors explore how New Yorkers seek medical advice and how
this can be predicted by other variables found in the NYC Community Health Survey.
The exercise demonstrates that the use of Naive-Bayes and Support Vector Machine (SVM)
classification techniques and their in performance in a real life scenario.
\end{abstract}

\section{Introduction}
The main goal of this assignment is to demonstrate the use of two classification
techniques, Naive-Bayes and Support Vector Machine (SVM), on predicting how
New Yorkers seek medical advice.

The authors examined the results of the Community Health Survey (CHS) conducted
by the New York City Department of Health and Mental Hygiene. This rich dataset
includes measurements on multiple aspects of health-- including access, nutrition,
demographic information, diet and exercise habits, medical history, and lifestyle
choices.

The motivation is to better understand how New Yorkers seek and receive advice
for their medical needs. Outreach programs hoping to improve access to health care
can make use of these results.

\section{Methods and Data Sets}

The raw dataset of the 2014 NYC Community Health Survey contains 8562 observations
across 188 variables. For this assignment, a subset of the raw dataset was used
for analysis, and the variables are described below:

\begin{itemize}
\item Sick Advice: A categorical variable about the respondent's usual resource for
health advice. This is the dependent variable in the analyses for this
assignment. The options are
``A private doctor",
``Community health center",
``A hospital outpatient clinic",
``ED/urgent care center",
``Alternative health care provider",
``Family/friend/self/resources",
``Non-hospital clinic",
``Other",
``No usual place", or
``Clinic, unknown type".
\item Education: A categorical variable on educational attainment. Categories are ``Less than HS", ``High school grad",
``Some college", or ``College graduate"
\item Marital status: A categorical variable on marital status. Categories are
``Married",
``Divorced",
``Widowed",
``Separated",
``Never married", or
``Member of unmarried couple living together"
\item US born: A binary variable answering the yes or no question "Are you US or foreign born?"
\item Sexual ID: A categorical variable on sexual orientation. Categories are
``Heterosexual",
``Gay/Lesbian", or
``Bisexual"
\item At Home Language: A categorical variable answering the question ``What language do you
speak most often at
home?" Options are
``English",
``Spanish",
``Russian",
``Chinese",
``Indian", or
``Other"
\item Insured: A binary variable answering the yes or no question ``Do you have any
kind of health
insurance coverage,
including private
health insurance,
prepaid plans such as
H-M-Os, or
government plans
such as Medicare or
Medicaid?"
\item The dependent variable being classified and predicted is `Sick Advice' in this assignment's
analyses. The authors refer to this variable as `Y'. The goal is to attempt to
predict how people will seek their medical advice based on other characteristics.
\item The independent variables are all the others.
\item  For all variables, the options ``do not know" and ``refuse"
were treated as null values and dropped.
\item Finally, independent variables are binarized so that all samples are represented
by boolean feature vectors.
\end{itemize}


The steps taken for this assignment:

\begin{enumerate}
\item A classification model is fitted for the binarized 6 independent variables with
a Naive Bayes algorithm assuming Bernoulli distributions.
\item A second classification model is fitted on the same variables with
a SVM using a linear kernel.
\item Finally, a third classification model is fitted with a SVM using a radial basis
function (RBR) kernel.
\item The authors adjust the RBF kernel parameters to assess how these parameters
affects the overall performance of the classification model and possible overfitting.
\item All models are assessed via cross validation on a training and test set. The training
and test set sizes are adjusted to assess the effect of training size on quality
of classification.
\item A confusion matrix is created for all categories of the dependent variable
``Sick Advice" to assess the performance of the classifications.
\item Finally, the model is used
\end{enumerate}


\section{Results}

\section{Conclusions}

In summary, the classification models performed poorly in predicting
how people tend to seek out medical advice based on other characteristics.

Further steps for this assignment would include earlier years of data from the
NYC Community Health Survey to expand the dataset, since we have seen that training
size of the dataset can have an affect on classification models.

Python code used to generate the results, tables, and figures for this assignment can be
found at \url{https://github.com/michellemho/machine_learning_for_cities}.

\begin{center}
\begin{table*}[]
\centering
\caption{Results of Classification Models}
\label{my-label}
\begin{tabular}{lllll}
                & Normalized raw data (LM1) &          & PCA-featured data (LM2) & \\
constant        & 0.40926343                & constant & 0.4101                  & \\
Neighborhood    & -0.12254491               & x1       & -0.0379                 & \\
BldClassif      & 0.02690248                & x2       & -0.0113                 & \\
YearBuilt       & 0.02695934                & x3       & -0.0441                 & \\
GrossSqFt       & 0.13819696                & x4       & -0.1773                 & \\
GrossIncomeSqFt & 0.01101943                & x5       & -0.0273                 & \\
R-squared       & 0.001                     & R-squared& 0.001                   & \\
                &                           &          &                         &
\end{tabular}
\end{table*}
\end{center}

\begin{figure*}[!t]
  \begin{center}
    \includegraphics[width=2.5in]{cusp-index.png}
  \end{center}

  \caption{\CUSP logo from website}
  \label{fig-1}
\end{figure*}



\end{document}
