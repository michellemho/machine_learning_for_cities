\documentclass[10pt,twocolumn]{article}
\usepackage{graphicx}
\usepackage{amssymb}
\usepackage{titling}
\usepackage{url,hyperref}

\begin{document}

\title{Classifying post-secondary school funding type based on student debt and
earnings}

\author{Shay Lehmann and Michelle Ho}

\date{%
CUSP-GX-5006\\
Prof. Gustavo Nonato\\
Machine Learning Assignment \# 3\\
\rule{\textwidth}{1pt}
}

\posttitle{\par\rule{3in}{0.4pt}\end{center}\vskip 0.5em}
%\postdate{\rule{\textwidth}{1pt}}

\maketitle

\begin{abstract}
In this assignment, the authors explore how 4-year colleges in the United States
can be classified into funding types (public, private for-profit, or private non-profit) based
on data on the debt and earning outcomes of their students.
The exercise aims to compare Support Vector Machines (SVM), decision trees (CART),
and Random Forests (with and without boosting) as
classification techniques and their in performance in a real life scenario.
\end{abstract}

\section{Introduction}
The main goals of this assignment are to compare three classification techniques and
to better understand how earning and debt outcomes of students vary by the types
post-secondary institutions they attend. The motivation is to ultimately understand
what indicates a "high value" education for the final project of this course.

The dataset used in this assignment is provided by the U.S. Department
of Education project, College Scorecard. The goal of the project is to provide data
necessary for students and their families to compare and assess
post-secondary institutions on their costs and student outcomes. The data is
compiled from self-reported data from institutions, data on federal financial aid, and
tax information for the past 20 years. This rich dataset
includes measurements on multiple aspects of a postsecondary institution-- including basic
identifying information, admissions, student body demographics, degree programs,
tuition costs, federal aid, debt repayment, completion rates, and earnings.

Our hope is that students and their families who are making decisions on
post-secondary education can make use of our results to compare schools.
Institutions themselves can use the results to assess how their students perform
compared to peer institutions. Finally, loan granting institutions and the U.S.
Department of Education can assess which schools are failing in preparing their
students for career success and why.

\section{Methods and Data Sets}

In order to have data on earnings and repayment outcomes after graduation,
this assignment uses only the 2010 - 2011 College Scorecard, since wage and debt data is not yet
available for the years after 2011. This dataset contains
7414 institutions and 1743 variables. However, there is not complete coverage for
all variables depending on the institution. For this assignment, subsets of the raw
dataset was used for analysis, and the variables and data cleaning steps are
described below:

\begin{itemize}
\item \texttt{CONTROL}: a categorical variable indicating the funding type of a school.
Categories are 1=``public", 2=``private non-profit", or 3=``private for-profit".
\item \texttt{REGION}: a categorical variable indicating the region of the United States
where the school is located. Regions are:
\begin{itemize}
\item New England (CT, ME, MA, NH, RI, VT)
\item Mid East (DE, DC, MD, NJ, NY, PA)
\item Great Lakes (IL, IN, MI, OH, WI)
\item Plains (IA, KS, MN, MO, NE, ND, SD)
\item Southeast (AL, AR, FL, GA, KY, LA, MS, NC, SC, TN, VA, WV)
\item Southwest (AZ, NM, OK, TX)
\item Rocky Mountains (CO, ID, MT, UT, WY)
\item Far West (AK, CA, HI, NV, OR, WA)
\item Outlying Areas (AS, FM, GU, MH, MP, PR, PW, VI)
\end{itemize}
\item \texttt{GRAD\char`_DEBT\char`_MDN}: Median debt for students who graduate
\item \texttt{WDRAW\char`_DEBT\char`_MDN}: Median debt of students who withdrew without completion
\item \texttt{GRAD\char`_DEBT\char`_MDN10YR}: Median debt by monthly payment (10 year plan) for graduates
\item \texttt{MN\char`_EARN\char`_WNE\char`_P7}: Mean earnings of students working and not enrolled 7 years
after entry
\item \texttt{COMPL\char`_RPY\char`_3YR\char`_RT\char`_SUPP}: 3-year repayment rate for students who completed degrees,
suppressed for institutions of fewer than 30 students
\item \texttt{NONCOM\char`_RPY\char`_3YR\char`_RT\char`_SUPP}: 3-year repayment rate for students who did not
complete degrees, suppressed for institutions of fewer than 30 students
\item \texttt{LO\char`_INC\char`_DEBT\char`_MDN}
\item \texttt{HI\char`_INC\char`_DEBT\char`_MDN}
\item \texttt{MD\char`_INC\char`_DEBT\char`_MDN}
\item \texttt{LOAN\char`_EVER}
\item The dependent variable being classified and predicted is `Control' in this assignment's
analyses. The authors may occasionally refer to this variable as `Y'.
\item The independent variables are all the others.
\item Only institutions classified as \texttt{ICLEVEL = 1} (eg. 4-year colleges) are used.
\item Any observations with null values for any of the chosen variables
are dropped. After these drops, the number of observations
in the dataset is 2249.
\item Finally, the categorical variable \texttt{REGION} is binarized so that all samples
are represented by boolean feature vectors.
\end{itemize}

The steps taken for this assignment:

\begin{enumerate}
\item A classification model is fitted for selected independent variables with
a SVM Linear model (Model 1)
\item A second SVM linear classification model is fitted using a second subset
of independent variables (Model 2)
\item Decision tree classifier models are fitted on the second subset of independent variables
\item The decision tree classifier models are done without boosting (Model 3), with bagging (Model 4),
with ADA boosting (Model 5), and with gradient boosting (Model 6) at varying depths.
\item All models are assessed via cross validation on training and test sets split
from the original dataset. The training and test set sizes are adjusted to assess the effect of training size on quality
of classification.
\item Confusion matrices are created for all categories of the dependent variable
``Control" to assess the performance of the classifications (Figures 1 - X)
\end{enumerate}

\section{Results}

The first classification model was SVM with a linear kernel, fitted on the variables \texttt{REGION},
\texttt{GRAD\char`_DEBT\char`_MDN}, \texttt{WDRAW\char`_DEBT\char`_MDN}, \texttt{GRAD\char`_DEBT\char`_MDN10YR}, \texttt{MN\char`_EARN\char`_WNE\char`_P7},
\texttt{COMPL\char`_RPY\char`_3YR\char`_RT\char`_SUPP}, \texttt{nd `NONCOM\char`_RPY\char`_3YR\char`_RT\char`_SUPP'. The dataset was split into
training sets of 20\%, 40\%, 60\%, and 80\% of the whole. The performance
was not very good, and misclassified approximately 50 percent of the data for all
training sizes (See Table 1)

\begin{table}[ht]
\caption{Percentage of Misclassification, Model 1} % title of Table
\centering % used for centering table
\begin{tabular}{c c c} % centered columns (4 columns)
\hline
Training Size Percent & In-Sample Error & Out-Sample Error \\ [0.5ex] % inserts table
%heading
\hline % inserts single horizontal line
20 & 49.88 & 52.56 \\ % inserting body of the table
40 & 49.27 & 54.0 \\
60 & 48.85 & 55.0 \\
80 & 49.03 & 51.11 \\ [1ex] % [1ex] adds vertical space
\hline %inserts single line
\end{tabular}
\label{table:model1}
\end{table}


\section{Conclusions}

Python code used to generate the results, tables, and figures for this assignment can be
found at \url{https://github.com/michellemho/machine_learning_for_cities}.

\begin{table*}[h]
\centering
\caption{Percentages of each school type}
\label{my-label}
\begin{tabular}{lllll}
& Type                & Percentage &\\
& Public              & 27.17            & \\
& Private Non-Profit  & 43.09    & \\
& Private For-profit  & 29.74  & \\
                &                           &
\end{tabular}
\end{table*}


\begin{figure*}[!t]
  \begin{center}
    \includegraphics[width=6in]{cusp-index.png}
  \end{center}

  \caption{\small CUSP Logo-- Figure Template}
  \label{fig-1}
\end{figure*}

\end{document}
