\documentclass[10pt,twocolumn]{article}
\usepackage{graphicx}
\usepackage{amssymb}
\usepackage{titling}
\usepackage{tabularx}
\usepackage{url,hyperref}
\pagenumbering{gobble}
\begin{document}

\title{Finding Communities within the Citibike Network}

\author{Shay Lehmann and Michelle Ho}

\date{%
CUSP-GX-5006\\
Prof. Gustavo Nonato\\
Machine Learning Assignment \# 4\\
\rule{\textwidth}{1pt}
}

\posttitle{\par\rule{3in}{0.4pt}\end{center}\vskip 0.5em}
%\postdate{\rule{\textwidth}{1pt}}

\maketitle

\begin{abstract}

In this assignment, the authors explore different clustering algorithms
as a way to find communities within the Citibike network. In this directed
network, the nodes are station locations and the edge weights are the count of trips
over the given time period of data. The clustering techniques used are K-means,
Spectral Clustering, and Hierarchical Clustering.

\end{abstract}

\section{Introduction}
Detecting communities within networks is one application of machine learning clustering
algorithms. From making recommendations within a  social networks to predicting the
spread of disease, there are many applications for community detection.


\section{Methods and Data Sets}

The dataset chosen for this assignment is Citibike trip data from July 2013 to February 2014,
as well as the station locations for that same time period. The variables available
from Citibike System Data are:

\begin{itemize}
\item `tripduration',
\item `starttime',
\item `stoptime',
\item `start station id',
\item `start station name',
\item `start station latitude',
\item `start station longitude',
\item `end station id',
\item `end station name',
\item `end station latitude',
\item `end station longitude',
\item `bikeid',
\item `usertype',
\item `birth year',
\item `gender'
\end{itemize}

The steps taken for this assignment:

\begin{enumerate}
\item A simple k-means algorithm is used to cluster the stations purely on station
location information
\item A directed network graph is created using the stations as nodes and the edges as trips
between stations. The edges are weighted by the sum of all trips that happened
between the stations during the study period.
\item The network graph is then used to generate spectral clusters. First, the
laplacian matrix is calculated using the python library networkx. Because the network
is directed, the default edge weight corresponds to outbound trips.
\item Hierarchal clusters are also generated based on the network graph
\item Parameters for the distance metric, number of clusters, and tree pruning are altered
\item The results are visualized on a map by coloring the station locations by their
community labels and with corresponding dendrograms for hierarchical clusters
\end{enumerate}

\section{Results}
The results show that


\section{Conclusions}

In summary

Python code used to generate the results, tables, and figures for this assignment can be
found at \url{https://github.com/michellemho/machine_learning_for_cities}.


\begin{figure*}[!t]
  \begin{center}
    \includegraphics[width=\textwidth,height=\textheight,keepaspectratio]{cusp-index.png}
  \end{center}

  \caption{\small Model 7}
  \label{fig-1}
\end{figure*}

\end{document}
