\documentclass[10pt,twocolumn]{article}
\usepackage{graphicx}
\usepackage{amssymb}
\usepackage{titling}
\usepackage{tabularx}
\usepackage{url,hyperref}
\pagenumbering{gobble}
\begin{document}

\title{Finding Communities within the Citibike Network}

\author{Shay Lehmann and Michelle Ho}

\date{%
CUSP-GX-5006\\
Prof. Gustavo Nonato\\
Machine Learning Assignment \# 4\\
\rule{\textwidth}{1pt}
}

\posttitle{\par\rule{3in}{0.4pt}\end{center}\vskip 0.5em}
%\postdate{\rule{\textwidth}{1pt}}

\maketitle

\begin{abstract}

In this assignment, the authors explore different clustering algorithms
as a way to find communities within the Citibike network. In this directed
network, the nodes are station locations and the edge weights are the count of trips
over the given time period of data. The clustering techniques used are K-means,
Spectral Clustering, and Hierarchical Clustering.

\end{abstract}

\section{Introduction}
Detecting communities within networks is one application of machine learning clustering
algorithms. From making recommendations within a  social networks to predicting the
spread of disease, there are many applications for community detection.


\section{Methods and Data Sets}

The dataset chosen for this assignment is Citibike trip data from July 2013 to February 2014,
as well as the station locations for that same time period. The variables available
from Citibike System Data are:

\begin{itemize}
\item `tripduration',
`starttime',
`stoptime',
`start station id',
`start station name',
`start station latitude',
`start station longitude',
`end station id',
`end station name',
`end station latitude',
`end station longitude',
`bikeid',
`usertype',
`birth year',
`gender'
\end{itemize}

The steps taken for this assignment:

\begin{enumerate}
\item A k-means clustering algorithm is used to cluster the stations based on
weighted distance matrix. The weighted distance matrix is calculated as the
shortest path of all stations to each other, weighted by the inverse of the
sum of trips between each station during the study period. The process
is repeated for 2 - 7 clusters (Figure 1).
\item A directed network graph is then created using the stations as nodes and the edges as trips
between stations. The edges are weighted by the same inverse sum of trips.
\item The network graph is then used to generate spectral clusters with a RBF (Gaussian)
kernel. First, the
laplacian matrix is calculated for the directed network using the python library networkx.
\item Then, the python library sci-kit learn is used to generate 2 - 7 spectral
clusters with the laplacian matrix (Figure 2)
\item Hierarchal clusters are also generated based on the network graph. First,
the shortest path length for all pairs in the network graph are calculated, weighting
the edges by the inverse weight. Smaller values indicate "closer" stations, in other words,
stations that have many trips between them.
These shortest path lengths are used to generate a condensed distance matrix in the
python library SciPy, and this matrix was finally used to generate hierarchy trees
with single and complete
linkage hierarchies using SciPy.
\item  Dendrograms are generated for the two hierarchy trees (Figure 3)
\item The hierarchies are visualized on a map by coloring the station locations by their
cluster labels for varying cuts of the trees that generate 2 - 7 clusters (Figures 4 and 5)
\item Special attention is paid to n clusters = 4 for the four clustering methods
(k-means, spectral, and hierarchy with single and complete linkage)
for discussion (Figure 6).
\end{enumerate}

\section{Results}
The results show that spectral clustering generates clusters within the network
graph have a discernible pattern that aligns somewhat with our expectations. Our expectation
was that the Citibike
network would show greater connectivity within boroughs. With spectral clustering in Figure 6,
we see a cluster in areas where people tend to work (cyan dots in
downtown and midtown Manhattan,
and downtown Brooklyn and parts of Williamsburg). Other clusters tend to fall where
there is higher residential population and along the east and west sides of
Manhattan (green and purple). A single station in red indicates a very unique cluster.
This station is situated between South Brooklyn, North Brooklyn, and Manhattan. Its
uniqueness in "pull" from all directions may have caused Spectral Clustering to cluster
this station on its own.

The single linkage hierarchical clustering produced the strangest results. Nearly
every station was classified into the same cluster, except for three unique stations
that were each in their own cluster. Looking at the dendrogram (Figure 3), this
corresponds to what is happening in the tree. Single linkage merges clusters
that minimize the distance between the clusters closest points. Complete linkage,
on the other hand, merges clusters that minimize the distances between the clusters
farthest points. With single linkage, a network graph can be quickly clustered together
because if several stations share a similar "trip profile".

The complete linkage hierarchical clustering and kmeans clustering appear to create
evenly spread clusters throughout the network, without any pattern to discern. Again,
this is probably because stations may share very similar "trip profiles".

\section{Conclusions}

In summary, spectral clustering yields results that aligned most with our expectations.
We were able to discern a pattern in how people used Citibike in each cluster based
on our knowledge of Manhattan and Brooklyn, and their corresponding residential and
commercial areas. However, the interpretation of these clusters is difficult given
that the Citibike network is not a typical social network.
Calculating the "shortest path" between nodes has no real-life interpretation for
Citibike. For example, if the shortest path between Stations A and B is through
a more popular station C, this tells us that A and B are not very directly connected, but
may fall into the same cluster anyway. Further examination of communities within
the Citibike network need to consider different distance metrics in addition
to the shortest path.

Python code used to generate the results and figures for this assignment can be
found at \url{https://github.com/michellemho/machine_learning_for_cities}.


\begin{figure*}[!t]
  \begin{center}
    \includegraphics[width=\textwidth,height=\textheight,keepaspectratio]
    {kmeans_weighted.png}
  \end{center}

  \caption{\small K-Means Clustering}
  \label{fig-1}
\end{figure*}

\begin{figure*}[!t]
  \begin{center}
    \includegraphics[width=\textwidth,height=\textheight,keepaspectratio]
    {spectralMMH.png}
  \end{center}

  \caption{\small Spectral Clustering}
  \label{fig-2}
\end{figure*}

\begin{figure*}[!t]
  \begin{center}
    \includegraphics[width=\textwidth,height=\textheight,keepaspectratio]
    {dendrograms.png}
  \end{center}

  \caption{\small Dendrograms}
  \label{fig-3}
\end{figure*}


\begin{figure*}[!t]
  \begin{center}
    \includegraphics[width=\textwidth,height=\textheight,keepaspectratio]
    {hier_weighted_singlelink.png}
  \end{center}

  \caption{\small Hierarchy Single-Linkage}
  \label{fig-4}
\end{figure*}


\begin{figure*}[!t]
  \begin{center}
    \includegraphics[width=\textwidth,height=\textheight,keepaspectratio]
    {hier_weighted_complete.png}
  \end{center}

  \caption{\small Hierarchy Complete-Linkage}
  \label{fig-5}
\end{figure*}


\begin{figure*}[!t]
  \begin{center}
    \includegraphics[width=\textwidth,height=\textheight,keepaspectratio]
    {CLUSTERS_OF_FOUR.png}
  \end{center}

  \caption{\small Multiple Clustering Methods}
  \label{fig-6}
\end{figure*}

\end{document}
